\lab{Installing and Updating Python Packages}{Installing and Updating Python Packages}
\label{updateinstall}

\objective{The process of installing and updating Python packages.}

\section*{Installing New Packages}
There are a number of ways to install Python packages. We suggest trying the following methods in order.

\subsection*{Anaconda Package Manager}
There are some packages that you can install through Anaconda that were not included by default in your initial installation.  
Look to see if the package you wish to download is found here: \url{http://docs.continuum.io/anaconda/pkg-docs.html}. If the package you wish to install is not included in this list, this method is not an option.
 
\begin{lstlisting}
$ conda update conda   # ensures you have the most recent version
$ conda install <packagename>  # Do not include < >
\end{lstlisting} 
You do not need to download anything extra, just run the command. It is not necessary to include the version number in this command. The package name is sufficient.


\subsection*{Pip}
\li{pip} is one of the most popular Python Package Managers and it works similarly to \li{conda}. Go to the following website to see if the package you would like to install is included: \url{https://pypi.python.org/pypi?\%3Aaction=index}. If the package you wish to install is not included in this list, this method will not be an option.

\begin{lstlisting}
$ pip install <packagename>  # Do not include < >
\end{lstlisting}
%$

\subsection*{Windows Only: Wheel Files}
Wheel files (\li{.whl}) are becoming more popular. Wheel is a built-package format, so you do not need to worry about compiling anything. Though this method will work with some extra effort on MacOSX and Linux, it is currently easier to perform on Windows. This is because wheel files for MacOSX and Linux are currently hard to find. Trying other methods may be a better use of your time. For Windows, \url{http://www.lfd.uci.edu/~gohlke/pythonlibs/} contains a very large list of packages for Python. You will notice that there are multiple versions of these packages available. Most will be in a format similar to the following examples:

\begin{lstlisting}
pandas-0.16.2-cp27-none-win_amd64.whl # 64-bit, Python 2.7
pandas-0.16.2-cp27-none-win32.whl # 32-bit, Python 2.7
\end{lstlisting}

Be sure to download the version compatible with your computer and Python 2.7. Once the wheel file is downloaded, navigate to the directory where the file is saved and run the following:

\begin{lstlisting}
$ pip install wheel   # Only need to do this once
$ pip install <filename>  # Do not include < >
\end{lstlisting}

\subsection*{If All Else Fails...}
If all else fails, you can download the package you need from the source. There are often installation instructions included in a package, but it can be difficult to set up everything correctly. If you are having trouble, it is likely others have had difficulties as well. A quick internet search may provide some direction.

\begin{info}
The best way to ensure a package has been installed correctly is to try importing it in a Python Interactive Shell.
\end{info}

\section*{Updating packages}

\begin{info}
If you update NumPy, you will need to modify the file \li{C:\\Anaconda\\Lib\\site-packages\\numpy\\distutils\\fcompiler\\gnu.py} as outlined in Appendix \ref{pythoninstall}.
\end{info}

\subsection*{Anaconda}
As a general rule, you update your packages using the same system you used to install them.
If you want to update a package included by using in Anaconda, you can run (in the command terminal) the command
\begin{lstlisting}
$ conda update <packagename>   # Do not include < >
\end{lstlisting}
%$

If you want to update all of the packages that are included in Anaconda, you can run the two commands
\begin{lstlisting}[style=ShellInput]
$ conda update conda
$ conda update anaconda
\end{lstlisting}
The first of these commands updates Anaconda's package manager.
The second updates all packages so that they match with the current release version of Anaconda. Because Anaconda has this capability, it is a good idea to install packages using \li{conda} whenever possible.

\subsection*{Pip}

If you installed a package via \li{pip}, you can upgrade it using the command
\begin{lstlisting}
$ pip install --upgrade <packagename>  # Do not include < >
\end{lstlisting}

If you would like to see the version number of all the packages you have installed, run:

\begin{lstlisting}[style=ShellInput]
$ pip freeze
\end{lstlisting}

\subsection*{Other}

If you installed a package via an online installer, get an updated version from
the same source and follow the same steps as you did when you first installed it.

\begin{warn}
Be careful not to try to update a Python package while it is in use.
It is safest to update packages while there is no Python interpreter currently running.
\end{warn}
