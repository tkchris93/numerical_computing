\lab{Optimal Reentry of a Spacecraft}{Optimal Reentry of a Spacecraft}
\label{lab:reentry}
\objective{
We look at two problems related to space flight: The second problem is to choose the optimal path for reentry into the atmosphere, where the total heating experienced by the craft is minimized. 
}


We begin by giving the control system that describes the path of the vehicle through the atmosphere
\footnote{This control problem and its numerical solution are thoroughly described in 'Introduction to Numerical Analysis' by J. Stoer, R. Bulirsch (pg 524). 
We will mirror their presentation throughout this lab.}. 
If $v$ is the velocity of the spacecraft, $\gamma$ is the angle of the flight path, $\xi$ is the normalized altitude above the Earth's surface ($\xi = h/R$, where $R$ is the radius of the Earth and $h$ the altitude of the spacecraft above the Earth), and $u$ is the control variable that represents the angle of attack of the spacecraft, the flight path can be described by 
\begin{align}
\begin{split}
v' &= -s\rho v^2C_D(u) - \frac{g\sin(\gamma)}{(1+\xi)^2},\\
\gamma ' &= s \rho v C_L(u) + \frac{v \cos(\gamma)}{R(1+\xi)} - \frac{g \cos \gamma}{v(1+\xi)^2},\\
\xi' &= \frac{v \sin \gamma}{R}.
\end{split} \label{eqn:reentry:control_system}
\end{align}
$C_D$ and $C_L$ represent drag and lift coefficients, and are functions of the angle of attack $u$: $C_D(u) = 1.174 - .9\cos u$, $C_L(u) = 0.6\sin u$. 
The atmospheric density is represented by $\rho = \rho_0e^{-R\beta\xi}$, where  $\rho_0$ is the atmospheric density at the surface of the earth. 
Parameter $s = \frac{1}{2}S/m$, where $S$ is the frontal area of the craft and $m$ is its mass, and $g$ represents the force of gravity. 
The numerical values we will use are $g = 3.2172\times10^{-4}$, $s = 26,600$, $R = 209$ $(209\times 10^5 \text{ ft})$, $\beta = 4.26$ and $\rho_0 = 2.704\times 10^{-3}$.
The trajectory of the spacecraft must also satisfy the boundary conditions
% \begin{align}
% \begin{split}
% 	v(0) &= 0.36 \quad (36000 \text{ ft/sec}),\\
% 	\gamma(0) &= -8.1^\circ \frac{\pi}{180^\circ}, \\
% 	\xi(0)&= \frac{4}{R}\quad (h = 400000 \text{ ft}), \\
% 	v(T) &= 0.27,\\
% 	\gamma(T) &= 0, \\
% 	\xi(T)&= \frac{2.5}{R}, \\
% \end{split}
% \end{align}
% 
\begin{equation}
  \begin{split}
    v(0) &= 0.36 \quad (36000 \text{ ft/sec})\\
    \gamma(0) &= -8.1^\circ \frac{\pi}{180^\circ}\\
    \xi(0)&= \frac{4}{R}\quad (h = 400000 \text{ ft})
  \end{split}
\quad \quad \quad \quad \quad
  \begin{split}
    v(T) &= 0.27\\
    \gamma(T) &= 0 \\
	\xi(T)&= \frac{2.5}{R}
  \end{split}
\end{equation}
where $T$ represents the time at the end of the reentry maneuver. 
To simpify notation, we will also write \eqref{eqn:reentry:control_system} in the form $y' = G(y)$, where $y = [y_0, y_1, y_2]^T=[v,\gamma, \xi]^T$ and $G$ has component functions $G = [G_0, G_1, G_2]^T$. 

We consider the problem of choosing the path of reentry for a spacecraft, so as to minimize the total heating of the craft. 
The functional describing the total heating is 
\[
J[u] = \int_0^T 10y_0^3 \sqrt{\rho}.
\]
The Hamiltonian for this control system is 
\begin{align}
H &=  10y_0^3 \sqrt{\rho} + \lambda_0G_0 + \lambda_1G_1 + \lambda_2G_2,
\end{align}
where $\lambda = [\lambda_0,\lambda_1,\lambda_2]^T$ is the adjoint variable. 
The optimal state and adjoint equations are thus given by 
\begin{align}
	\dot{y} &= H_{\lambda},\\
	\dot{\lambda} &= -H_{y} \label{eqn:reentry:adjoint_system}
\end{align}
To our boundary conditions we add the terminal condition that $H = 0$ at $t = T$.  From the condition $\frac{\partial H}{\partial u} = 0$ we find that the optimal control satisfies
\begin{align}
\tan u &= \frac{6\lambda_1}{9y_0\lambda_0}.
\end{align}
% \footnote{BNDSCO - A program for the numerical solution of optimal control problems, H.J. Oberle and W. Grimm, 1989}

\begin{figure}
\centering
\includegraphics[width=\textwidth]{solutions.pdf}
\caption{The solution of ?%\eqref{reentry:solutions}.
}
\label{fig:reentry:solutions}
\end{figure}



\section*{Constructing an Initial Guess}
Most BVP solvers require an equal number of differential equations and boundary conditions. 
Currently we have 6 ODEs and 7 boundary conditions. 

This nonlinear BVP is very sensitive, and requires an initial guess that is quite close to the solution.  This is intuitively obvious: if the control 


Since this is a sensitive problem, we will use a heuristic method to help us find a good initial guess.


TODO: 
\begin{enumerate}
	\item Justify terminal condition $H = 0$. Check Jared's notes.
	\item Transform the free boundary value problem so that there are 7 ODEs and 7 BCs. 
	\item Use the control $u$ to help guess the adjoint trajectories
	\item Turn scattering paper into a lab to replace lab on Poisson's equation. 
	
\end{enumerate}
Guess that $u = p_0\erf(p_1(p_2-t/T))$, where $p_0, p_1,$ and $p_2$ are unknown constants. 
We define an auxiliary problem to help us come up with a good initial guess for the original BVP. 
\begin{align}
\begin{split}
y_0' &= -s\rho y_0^2C_D(u) - \frac{g\sin(y_1)}{(1+y_2)^2},\\
y_1' &= s \rho y_0 C_L(u) + \frac{y_0 \cos(y_1)}{R(1+y_2)} - \frac{g \cos y_1}{y_0(1+y_2)^2},\\
y_2' &= \frac{y_0 \sin y_1}{R} ,\\
p_0' &= 0, \\
p_1' &= 0, \\
p_2' &= 0.
\end{split} \label{eqn:reentry:control_system_auxiliary}
\end{align}



The following code includes import statements, defines the parameters of the BVP, and creates functions for the drag coefficients. 
\begin{lstlisting}
from __future__ import division
from math import sin, cos
from scipy.special import erf
from scikits import bvp_solver 

R = 209
beta = 4.26
rho0 = 2.704e-3
g = 3.2172e-4
s = 26600	
T_init = 230

def C_d(u): 
	return 1.174 - 0.9*cos(u)

def C_l(u): 
	return 0.6*sin(u)
	
\end{lstlisting}

For the auxiliary problem, we code functions for the ode and for the boundary conditions. 
\begin{lstlisting}
def ode_auxiliary(x,y):
	u = y[3]*erf( y[4]*(y[5]-x/T_init) )
	rho = rho0*exp(-beta*R*y[2])
	out = array([-s*rho*y[0]**2*C_d(u) - g*sin(y[1])/(1+y[2])**2,
				  ( s*rho*y[0]*C_l(u) + y[0]*cos(y[1])/(R*(1 + y[2])) - 
				  g*cos(y[1])/(y[0]*(1+y[2])**2) ),
				  y[0]*sin(y[1])/R,
				  0,
				  0,
				  0		])
	return out
	
def bcs_auxiliary(ya,yb):
	# If the problem is defined on the interval [a,b], ya = y(a) and yb = y(b), 
	# ya[0] = y_0(a), etc.
	# bvp_solver will attempt to make each of the entries below zero in the 
	# numerical solution.
	
	out1 = array([ya[0]-.36,
				  ya[1]+8.1*pi/180,
				  ya[2]-4/R
				  ])
	out2 = array([yb[0]-.27,
				  yb[1],
				  yb[2]-2.5/R
				  ])
	return out1, out2

problem_auxiliary = bvp_solver.ProblemDefinition(num_ODE = 6,
										  num_parameters = 0,
										  num_left_boundary_conditions = 3,
										  boundary_points = (0., T_init),
										  function = ode_auxiliary,
										  boundary_conditions = bcs_auxiliary)
									
solution_auxiliary = bvp_solver.solve(problem_auxiliary,
								solution_guess = guess_auxiliary)
								
N = 240 # Number of subintervals
x_array = linspace(0,T_init,N+1)
y_array = solution_auxiliary(x_array)
	
\end{lstlisting}

\begin{figure}
\begin{minipage}[b]{.47\linewidth}
\centering
\includegraphics[width=\textwidth]{u_heuristic.pdf}
\caption*{Heuristic for the control $u$, provided by engineers. }
\end{minipage}
\hspace{0.5cm}
\begin{minipage}[b]{0.47\linewidth}
\centering
\includegraphics[width=\textwidth]{u_heuristic_smooth.pdf}
\caption*{Providing a smooth initial guess for the control $u$.}
\end{minipage}
\caption{We attempt to find a smooth estimate for the control $u$, by giving the control the form 
$u = p_0\erf(p_1(p_2-t/T))$ and estimating parameters $p_0, p_1, p_2$.}
\label{fig:reentry:estimate_u}
\end{figure}





% \begin{figure}
% \begin{minipage}[b]{.47\linewidth}
% \centering
% \includegraphics[width=\textwidth]{diffusion_denoised_baloons_resized_bw.jpg}
% \caption*{Initial diffusion-based approach}
% \end{minipage}
% \hspace{0.5cm}
% \begin{minipage}[b]{0.47\linewidth}
% \centering
% \includegraphics[width=\textwidth]{tv_denoised_baloons_resized_bw.jpg}
% \caption*{Total variation based approach}
% \end{minipage}
% \caption{The solutions of \eqref{tv_images:diffusion_flow} and \eqref{tv_images:tv_flow}, found using a first order Euler step in time and centered differences in space.}
% \label{fig:noise_compare_attempts}
% \end{figure}



% \begin{align}
% \begin{split}
% \end{split} \label{reentry:label}
% \end{align}
%
%
% \begin{problem}
%
% \begin{lstlisting}
% \end{lstlisting}
% \end{problem}


% \begin{figure}
% \centering
% \includegraphics[width=6cm]{tv_denoised_baloons_resized_bw.jpg}
% \caption{The solution of \eqref{tv_images:diffusion_flow}, found using a first order Euler step in time and centered differences in space.}
% \label{fig:tv_image_denoised}
% \end{figure}







